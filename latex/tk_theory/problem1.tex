\setlength{\parindent}{0pt}\textbf{\huge Task 1}\\ 

1) advertisement
\begin{center}
\begin{tabular}{|l|l|l|l|}
\hline
Source & Destination & Request & Filter \\ \hline
1      & 3           & adv     & F      \\ \hline
3      & 2           & adv     & F      \\ \hline
3      & 4           & adv     & F      \\ \hline
4      & 5           & adv     & F      \\ \hline
4      & 6           & adv     & F      \\ \hline
\end{tabular}
\end{center}

2) subscription
\begin{center}
\begin{tabular}{|l|l|l|l|}
\hline
Source & Destination & Request & Filter \\ \hline
5      & 4           & sub     & F      \\ \hline
4      & 3           & sub     & F      \\ \hline
3      & 1           & sub     & F      \\ \hline
6      & 4           & sub     & F      \\ \hline
\end{tabular}
\end{center}

3) notification
\begin{center}
\begin{tabular}{|l|l|l|}
\hline
Source & Destination & Message                    \\ \hline
1      & 3           & notification n (n match F) \\ \hline
3      & 4           & notification n (n match F) \\ \hline
4      & 5           & notification n (n match F) \\ \hline
4      & 6           & notification n (n match F) \\ \hline
\end{tabular}
\end{center}

\setlength{\parindent}{0pt}\textbf{\huge Task 2}\\

1) In the context of a matchmaking system, it is very typical that the number of players (subscriber) outweighs the number of matchmaking servers (publisher). In such a scenario, each subscription from a subscription routing scheme will have to traverse the whole topology to establish routing tables on each network node. This will result in a massive network traffic and make the system not able to scale. However, after the traversing period, matchmaking messages (notification) could reach players soon enough based on the routing information previously constructed. While in routing with advertisement, matchmaking messages will only be able to reach players after the advertising phase and subscribing phase such that there is a longer duration for establishing routing network. Since only matchmaking servers are traversing the network for advertising, the number of traffic generated are notably lower comparing to subscription routing thus making it more scalable.\\

2) Restricted to its scalability, subscription routing is only suitable for system with less amount of subscribers as discussed above. In addition, subscription routing has an advantage as compared to advertisement routing when subscriptions filters (content based) shows a variety over value and structure since that in advertisement routing routes are activated based on predefined publication filters.\\
On the other hand, advertisement routing is capable of handling system with large amount of subscription and unsubscription and relatively less and unchanging publisher. This is due to its light-weighted routing table which has promoted the routing efficiency of the system.\\

%\setlength{\parindent}{0pt}\textbf{\huge Task 3}\\

%\setlength{\parindent}{0pt}\textbf{\huge Task 4}\\

%\setlength{\parindent}{0pt}\textbf{\huge Task 5}\\
